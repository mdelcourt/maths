\documentclass[main.tex]{subfiles}
\begin{document}

\tableofcontents

\chapter{Dérivées}

\begin{definition}
    [Dérivée]

    \subsubsection*{Cadre}

    \begin{itemize}
        \item $f$ est une fonction;
        \item $x \in \cl {\dom f}$ est un point \emph{non-isol\'e} de $\dom f$.
    \end{itemize}

    \subsubsection*{Définition}

    Si la limite
    \begin{align}
        \lim_{h \to 0} \frac {f(x + h) - f(x)} h
    \end{align}
    existe et est réelle,
    alors le nombre
    \begin{align}
        f'(x) \defeq \lim_{h \to 0} \frac {f(x + h) - f(x)} h
    \end{align}
    est appelé la \emph{dérivée} de $f$ en $x$.
\end{definition}

\end{document}
