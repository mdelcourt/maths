\documentclass[main.tex]{subfiles}
\begin{document}

\chapter{Intégrales}

\section{Introduction}

Historiquement,
le calcul intégral trouve ses origines dans de simples calculs d'aire ou de volumes de tonneaux de vin.
Durant la deuxième partie du XVII\textsuperscript{è} siècle,
l'intégrale est promue au rang d'outil fondamental de l'analyse et de la physique.
Elle fait désormais partie intégrante du langage des lois de la nature,
mais surtout elle est l'unique outil apte à les \emph{résoudre}.

Les évolutions scientifiques et technologiques au fil des siècles
renforceront la place de l'intégrale comme outil mathématique principal des scientifiques et des ingénieurs.
L'intégrale est utilisée pour le décollage de fusées, pour faire voler les avions,
mais également dans les manipulations de son et d'images.
Les applications sont infinies et font de ce chapitre l'apothéose du cours de mathématiques en secondaire.

L'intégrale a également eu une influence considérable sur les mathématiques.
Des domaines entiers (en particulier la théorie des probabilités et l'analyse harmonique)
ont été paralysés en attendant des progrès indispensables en théorie de l'intégration,
alors que ces domaines étaient eux-mêmes cruciaux pour les sciences.
De plus, l'intégrale permet de définir une notion convenable de distance entre fonctions,
ce qui s'avère être extrêmement important pour la résolution d'équations scientifiques.

Ce chapitre introduit également ce qui peut être considéré comme l'un des plus beaux théorèmes mathématiques,
le \emph{théorème fondamental du calcul différentiel et intégral}.
Dans le cadre de ce cours,
ses conséquences ne seront malheureusement réduites qu'aux calculs d'aire et de volume.
Il est cependant utile de garder à l'esprit l'importance théorique considérable de ce résultat,
qui fournit un outil pour étudier la solubilité des équations scientifiques,
et permet de généraliser la définition de dérivée.

Cependant, les programmes sont écrits par des mathématiciens.
Faisons donc comme s'il n'y avait pas eu d'avancée scientifique ou technologique depuis le XVII\textsuperscript{è} siècle,
et attelons-nous avec enthousiasme à résoudre des problèmes datant de l'Antiquité.
Le seul anachronisme autorisé est l'utilisation du terme \emph{nombre réel}.

\section{Vers l'intégrale de Cauchy-Riemann}

Supposons que nous cherchons à calculer l'aire entre une courbe et l'axe des abscisses
de la fonction dont le graphe est esquissé ci-dessous.
\begin{plot}{0.5}{-1}{-1}{7}{7}
    \plotfunction{1:6}{2*sin(\x r) + 4}
\end{plot}

L'idée est d'obtenir une approximation en approximant l'aire sous la courbe
par $n$ rectangles de même base.
\begin{plot}{0.5}{-2}{-1}{7}{7}
    \drawriemannsums{2*sin(\x r) + 4}{1}{6}{10}
    \def\c{3.25}
    \pgfmathsetmacro{\Vc}{2*sin(\c r)+4}
    \draw[dashed] (\c, 0) node[below] {\footnotesize{$x_j$}} -- (\c, \Vc) -- (0, \Vc) node[left] {\footnotesize{$f(x_j)$}};
    \filldraw[black] (\c, \Vc) circle (.05cm);
\end{plot}
Les bases de ces rectangles déterminent des sous-intervalles $I_1, \dots, I_n$ de longueur $\Delta x$.
La hauteur de chaque rectangle est déterminée par l'image $f(x_j)$ d'un point $x_j \in I_j$,
de telle sorte que la somme des aires des rectangles est
\begin{align}
    \sum_{j = 1}^n f(x_j) \Delta x.
\end{align}

L'idée est qu'en augmentant le nombre $n$ de rectangles,
la somme ci-dessus convergera vers l'\emph{aire sous la courbe}.
\begin{plot}{0.5}{-2}{-1}{7}{7}
    \drawriemannsums{2*sin(\x r) + 4}{1}{6}{40}
\end{plot}
En des termes mathématiques,
l'aire sous la courbe est donnée par
\begin{align}
    \lim_{n \to \infty} \sum_{j = 1}^n f(x_j) \Delta x.
\end{align}

\section{Définition}

Inspirés par la discussion de la section précédente,
nous introduisons la notion d'\emph{intégrale} sur un intervalle,
qui correspond intuitivement (pour une fonction positive) à un calcul d'aire sous la courbe.

Cette définition est généralement attribuée à Riemann,
mais Cauchy fut le premier à l'énoncer dans le cas d'une \emph{fonction continue}.
Dans ce chapitre,
nous travaillerons toujours avec de telles fonctions.

\begin{definition}
    [Intégrale de Cauchy]

    Soit $f$ une fonction continue sur $\ccinterval a b$ avec $a < b$.

    \begin{itemize}
        \item $\ccinterval a b$ est sudvidisé en $n$ sous-intervalles de même longueur $\Delta x$;
        \item $x_j$ appartient au $i$\textsuperscript{è} sous-intervalle pour chaque $j = 1, \dots, n$.
    \end{itemize}

    L'\emph{intégrale} de $f$ sur $\ccinterval a b$ est définie comme étant le nombre réel
    \begin{align}
        \int_a^b f(x) \dd x \defeq \lim_{n \to \infty} \sum_{j = 1}^n f(x_j) \Delta x.
        \label{eq:definition:integrale}
    \end{align}
\end{definition}

\begin{remark}
    [Existence de la limite et choix des $x_j$]

    L'existence de la limite~\eqref{eq:definition:integrale} n'est pas triviale et dépasse le cadre de ce cours.
    Pour illustrer la subtilité de ce résultat,
    notons que cette limite n'existerait pas si nous travaillions uniquement sur les nombres rationnels.
    Il faut également montrer que cette limite est indépendante du choix des $x_j$, $x = 1, \dots, n$.
\end{remark}

\begin{remark}
    [Interprétation de l'intégrale]

    \begin{itemize}
        \item Pour une fonction positive,
            l'intégrale $\int_a^b f(x) \dd x$ représente l'aire sous la courbe entre $a$ et $b$.
        \item Pour une fonction négative,
            l'intégrale $\int_a^b f(x) \dd x$ représente l'\emph{opposé} de l'aire au-dessus de la courbe entre $a$ et $b$.
            En effet,
            on remarque dans l'image ci-dessous
            \begin{plot}{0.5}{-2}{-7}{7}{1}
                \drawriemannsums{-2*sin(\x r) - 4}{1}{6}{10}
                \def\c{3.25}
                \pgfmathsetmacro{\Vc}{-2*sin(\c r)-4}
                \draw[dashed] (\c, 0) node[above] {\footnotesize{$x_j$}} -- (\c, \Vc) -- (0, \Vc) node[left] {\footnotesize{$f(x_j)$}};
                \filldraw[black] (\c, \Vc) circle (.05cm);
            \end{plot}
            que chaque $f(x_j)$, puisque négatif, représente l'\emph{opposé} de la hauteur du rectangle,
            rendant dès lors chacun des termes de la somme
            \begin{align}
                \sum_{j = 1}^n f(x_j) \Delta x
            \end{align}
            négatifs.
        \item Pour une fonction quelconque
            l'intégrale $\int_a^b f(x) \dd x$ représente l'aire \emph{signée} sous la courbe entre $a$ et $b$.
    \end{itemize}
\end{remark}

\begin{remark}
    [Pourquoi l'aire signée?]

    Une question légitime serait de se demander pourquoi~\eqref{eq:definition:integrale} ne contient pas de valeur absolue
    pour s'assurer que chaque contribution soit positive
    et que l'on mesure bien des aires de rectangle.

    \begin{itemize}
        \item L'aire \emph{signée} est nettement plus facile à calculer.
            En effet, la ressemblance formelle entre
            \begin{align}
                \int_a^b f(x) \dd x
                \quad \text{et} \quad
                \int f(x) \dd x
            \end{align}
            suggère l'existence d'un \emph{lien entre intégrale et primitive}.
            Ce lien est décrit par le \emph{théorème fondamental de l'analyse},
            et simplifie considérablement le calcul d'intégrales.

            Cependant, pour qu'un lien entre intégrale et primitive soit possible,
            il faut que ces deux notions se comportent de la même façon.
            En particulier,
            un changement de signe d'une fonction
            change le signe de ses primitives puisque
            \begin{align}
                \int -f(x) \dd x = - \int f(x) \dd x.
            \end{align}
            Pour qu'une telle propriété soit valide pour l'intégrale,
            il faut mesurer l'aire signée.
        \item Le calcul d'aire est très loin d'être l'unique application du calcul intégral.
            Certaines des applications les plus importantes
            (citons à titre d'exemple le calcul de moyenne, de centre de masse, de déplacement et de travail)
            calculent des nombres qui peuvent également être négatifs.
            En mesurant l'aire signée,
            on permet à l'intégrale d'être suffisament générale pour traiter des applications nettement plus vastes.
    \end{itemize}
\end{remark}

\section{Théorème fondamentaux du calcul différentiel et intégral}

Les théorèmes fondamentaux du calcul différentiel et intégral décrivent le comportement entre la dérivée et l'intégrale:

\begin{itemize}
    \item La partie I décrit ce qui se passe lorque l'on \emph{intègre une dérivée}.
    \item La partie II décrit ce qui se passe lorsque l'on \emph{dérive une intégrale}.
\end{itemize}

\begin{theorem}
    [Théorème fondamental de l'analyse --- Partie I]

    Soit $f$ une fonction dérivable sur $\ccinterval a b$.
    Si $f'$ est continue,
    alors on a
    \begin{align*}
        \int_a^b f'(x) \dd x = f(b) - f(a).
    \end{align*}
\end{theorem}
\begin{proof}
    Divisons $\ccinterval a b$ en $n$ intervalles $I_j = \ccinterval {a_j} {a_{j + 1}}$ de longueurs égales $\Delta x$.
    On remarque que
    \begin{align}
        f(b) - f(a)
        &= f(a_{n + 1}) - f(a_1)\\
        &= f(a_{n + 1}) - f(a_n) + f(a_n) - f(a_{n - 1}) + \cdots + f(a_2) - f(a_1)\\
        &= \sum_{j = 1}^n (f(a_{j + 1}) - f(a_j)).
    \end{align}

    En se souvenant que $a_{j + 1} - a_j = \Delta x$,
    on obtient
    \begin{align}
        f(b) - f(a)
        &= \sum_{j = 1}^n \frac {f(a_{j + 1}) - f(a_j)} {a_{j + 1} - a_j} \Delta x.
    \end{align}

    Par le théorème des accroissements finis,
    nous pouvons choisir $x_j \in \ccinterval {a_j} {a_{j + 1}}$, $j = 1, \dots, n$ tels que
    \begin{align}
        \frac {f(a_{j + 1}) - f(a_j)} {a_{j + 1} - a_j} = f'(x_j).
    \end{align}

    Dès lors, on a
    \begin{align}
        f(b) - f(a)
        &= \sum_{j = 1}^n f'(x_j) \Delta x.
    \end{align}

    En faisant tendre $n$ vers l'infini des deux côtés,
    on obtient
    \begin{align}
        f(b) - f(a) = \int_a^b f'(x) \dd x.
    \end{align}
\end{proof}

\begin{theorem}
    [Théorème fondamental de l'analyse --- Partie II]

    Soit $f$ une fonction continue sur $\ccinterval a b$.
    Alors la fonction définie par
    \begin{align*}
        F(x) \defeq \int_a^x f(t) \dd t
    \end{align*}
    est une primitive de $f$.
\end{theorem}

\section{Applications du calcul intégral}

\subsection{Calcul d'aire entre deux courbes}

\subsection{Volumes de solides de révolution}

Cherchons à calculer le volume obtenu par révolution du graphe autour de l'axe des abscisses.
Nous approximons le volume par des cylindres comme le montre la figure ci-dessous:
\begin{plot}
    {0.5}{-2}{-6}{12}{6}
    \drawcylinders{sqrt(2*\x + 1)}{1}{11}{10}
    \def\c{3.5}
    \pgfmathsetmacro{\Vc}{sqrt(2*\c + 1)}
    \draw[dashed] (\c, 0) node[below] {\footnotesize{$x_j$}} -- (\c, \Vc) -- (0, \Vc) node[left] {\footnotesize{$f(x_j)$}};
    \filldraw[black] (\c, \Vc) circle (.05cm);
\end{plot}

En se souvenant que le volume d'un cylindre de rayon $r$ et de hauteur $h$ est donné par $\pi r^2 h$,
nous en déduisons que la somme des volumes des cylindres est donnée par
\begin{align}
    \sum_{j = 1}^n \pi f^2(x_j) \Delta x.
\end{align}

Comme pour l'aire sous la courbe,
nous nous rapprochons du volume de révolution si le nombre de cylindre $n$ augmente.
\begin{plot}
    {0.5}{-1}{-6}{12}{6}
    \drawcylinders{sqrt(2*\x + 1)}{1}{11}{40}
\end{plot}

Nous en concluons que le volume de révolution est donné par
\begin{align}
    \lim_{n \to \infty} \sum_{j = 1}^n \pi f^2(x) \Delta x
    = \pi \int_a^b f^2(x) \dd x.
\end{align}

\begin{example}
    [Volume de la boule]

    La boule de rayon $r$ est obtenue par révolution de la fonction
    \begin{align}
        f(x) = \sqrt {r^2 - x^2}
    \end{align}
    autour de l'axe des abscisses sur $\ccinterval {-r} r$.
    \begin{plot}
        {1}{-3}{-3}{3}{3}
        \drawcylinders{sqrt(4 - (\x)^2)}{-2}{2}{10}
    \end{plot}

    Le volume de la boule est alors donné par
    \begin{align}
        V &= \pi \int_{-r}^r r^2 - x^2 \dd x\\
          &= 2 \pi \int_0^r r^2 - x^2 \dd x\\
          &= 2 \pi \left[r^2 x - \frac {x^3} 3\right]^r_0.
    \end{align}

    On en conclut alors que
    \begin{align}
        V = 2 \pi \left(r^3 - \frac {r^3} 3\right) = \frac 4 3 \pi r^3.
    \end{align}
\end{example}

\begin{example}
    [Volume du cône]

    Le cône de hauteur $h > 0$ ayant une base de rayon $r > 0$ est obtenu par révolution de la fonction
    \begin{align}
        f(x) = \frac r h x
    \end{align}
    autour de l'axe des abscisses sur $\ccinterval 0 h$.
    \begin{plot}
        {0.5}{-1}{-3}{5}{3}
        \drawcylinders{0.5*\x}{0}{4}{10}
    \end{plot}

    Le volume du cône est donné par
    \begin{align}
        V
        &= \frac {\pi r^2} {h^2} \int_0^h x^2 \dd x\\
        &= \frac {\pi r^2} {h^2} \frac {h^3} 3
        = \frac 1 3 \pi r^2 h
    \end{align}
\end{example}

\subsection{Valeur moyenne d'une fonction}

Supposons que nous voulions calculer la valeur moyenne $\mu$ d'une fonction continue $f$ sur un intervalle $\ccinterval a b$.

Une première approximation consisterait à choisir $n + 1$ points équidistants $x_1 = a$, $x_2$, \dots, $x_{n + 1} = b$
et de calculer
\begin{align}
    \mu \approx \frac 1 n \sum_{j = 1}^n f(x_j).
\end{align}

Notre choix de $n + 1$ points découpe $\ccinterval a b$ en $n$ intervalles de même longueur $\Delta x = \frac {b - a} n$.
Dès lors,
la valeur moyenne de $f$ sur $\ccinterval a b$,
peut être approximée par
\begin{align}
    \mu \approx \frac 1 n \sum_{j = 1}^n f(x_j)
    = \frac 1 {b - a} \sum_{j = 1}^n f(x_j) \Delta x.
\end{align}

Puisque l'approximation s'améliore au plus $n$ est grand,
on s'attend alors à ce que
\begin{align}
    \mu &= \lim_{n \to \infty} \left(\frac 1 {b - a} \sum_{j = 1}^n f(x_j) \Delta x\right)\\
        &= \frac 1 {b - a} \int_a^b f(x) \dd x.
\end{align}

\begin{definition}
    [Valeur moyenne d'une fonction continue]

    Soit $f$ une fonction continue sur un intervalle $\ccinterval a b$.
    La \emph{moyenne} de $f$ sur $\ccinterval a b$ est définie par
    \begin{align}
        \mu_{f, \ccinterval a b} \defeq \frac 1 {b - a} \int_a^b f(x) \dd x.
    \end{align}
\end{definition}

\begin{example}
    [Moyenne d'une fonction]

    Supposons que nous cherchions à calculer la valeur moyenne $\mu$ de $f(x) = x^2$ sur $\ccinterval 0 3$.

    Nous savons qu'elle vaut
    \begin{align}
        \mu = \frac 1 3 \int_0^3 x^2 \dd x = \frac 1 3 \cdot \frac {3^3} 3 = 3.
    \end{align}

    Cela peut se vérifier graphiquement:
    \begin{plot}{0.5}{-1}{-1}{4}{10}
        \plotfunction{0:3}{(\x)^2}
        \drawline (0, 3) node[left] {\footnotesize{$\mu$}} -- (3, 3);
    \end{plot}
\end{example}

\end{document}
