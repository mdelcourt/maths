\documentclass[main.tex]{subfiles}
\begin{document}

\chapter{Intégrales}

\section{Définition}

\begin{definition}
    [Intégrale de Cauchy]

    Soit $f$ une fonction continue sur $\ccinterval a b$ avec $a < b$.

    \begin{itemize}
        \item $\ccinterval a b$ est sudvidisé en $n$ sous-intervalles de même longueur $\Delta x$;
        \item $x_j$ appartient au $i$\textsuperscript{è} sous-intervalle pour chaque $j = 1, \dots, n$.
    \end{itemize}

    L'\emph{intégrale} de $f$ sur $\ccinterval a b$ est définie comme étant le nombre réel
    \begin{align}
        \int_a^b f(x) \dd x \defeq \lim_{n \to \infty} \sum_{j = 1}^n f(x_j) \Delta x.
    \end{align}
\end{definition}

\begin{remark}
    [Interprétation de l'intégrale]

    \begin{itemize}
        \item Pour une fonction positive,
            l'intégrale $\int_a^b f(x) \dd x$ représente l'aire sous la courbe entre $a$ et $b$.
        \item Pour une fonction négative,
            l'intégrale $\int_a^b f(x) \dd x$ représente l'\emph{opposé} de l'aire sous la courbe entre $a$ et $b$.
        \item Pour une fonction quelconque
            l'intégrale $\int_a^b f(x) \dd x$ représente l'aire \emph{signée} sous la courbe entre $a$ et $b$.
    \end{itemize}
\end{remark}

\end{document}
