\documentclass[main.tex]{subfiles}
\begin{document}
\chapter{Primitives}

\section{Synthèse}

\begin{proposition}
    [Primitivation par parties]

    \subsubsection{Hypothèses}
    \begin{itemize}
        \item $f, g$ deux fonctions dérivables sur un intervalle $I$.
    \end{itemize}

    \subsubsection{Thèses}
    \begin{itemize}
        \item $f' g$ est primitivable sur $I$ ssi $f g'$ est primitivable sur $I$.
        \item On a alors
            \begin{align}
                \int f'(x) g(x) \dd x
                = f(x) g(x)
                - \int f(x) g'(x) \dd x.
            \end{align}
    \end{itemize}
\end{proposition}
\begin{proof}
    Partons de la formule de dérivation d'un produit
    \begin{align*}
        (f g)'(x) = f'(x) g(x) + f(x) g'(x).
    \end{align*}

    En réorganisant,
    nous obtenons
    \begin{align*}
        f'(x) g(x) = (f g)'(x) - f(x) g'(x).
    \end{align*}

    En primitivant les deux côtés,
    on obtient
    \begin{align}
        \int f'(x) g(x) \dd x
        &= \int (f g)'(x) \dd x - \int f(x) g'(x) \dd x\\
        &= f(x) g(x) - \int f(x) g'(x) \dd x.
    \end{align}
\end{proof}

\begin{proposition}
    [Changement de variable]

    \subsubsection{Hypothèses}
    \begin{itemize}
        \item $u$ est dérivable sur un intervalle $I$.
        \item $f$ est primitivable sur $u(I)$.
    \end{itemize}

    \subsubsection{Thèses}
    \begin{itemize}
        \item $(f \circ u) u'$ est primitivable sur $I$;
        \item On a
            \begin{align}
                \int f(u(x)) u'(x) \dd x
                = F(u(x)) + C,
            \end{align}
            où $F$ est une primitive de $f$ sur $u(I)$.
    \end{itemize}
\end{proposition}
\begin{proof}
    Soit $F$ une primitive de $f$.
    En partant du membre de droite,
    et en se souvenant que la primitive est l'\emph{inverse} de la dérivée,
    on a
    \begin{align}
        F(u(x)) + C
        &= \int (F(u(x)) + C)' \dd x\\
        &= \int F'(u(x)) u'(x) \dd x,
    \end{align}
    où la deuxième ligne a été obtenue par la dérivée de fonction composée.

    Puisque $F$ est une primitive de $f$, $F' = f$ et donc
    \begin{align}
        F(u(x)) + C
        &= \int f(u(x)) u'(x) \dd x,
    \end{align}
    ce qui était bien le résultat à prouver.
\end{proof}

\end{document}
