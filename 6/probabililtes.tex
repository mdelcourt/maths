\documentclass[main.tex]{subfiles}
\begin{document}

\chapter{Probabilités}

\section{Introduction}

\section{Probabilité discrète}

\begin{definition}
    [Expérience aléatoire]

    Une \emph{expérience aléatoire} est une expérience
    dont on ne peut prédire l'issue finale.
\end{definition}

\begin{definition}
    [Probabilité (discrète)]

    \subsubsection{Cadre}
    \begin{itemize}
        \item $\Omega$ désigne toutes les issues possible d'une expérience aléatoire.
    \end{itemize}

    \subsubsection{Définition}
    Une probabilité est une fonction qui à chaque \emph{issue} $I \in \Omega$
    associe un nombre $\Pr(I)$ dans $\ccinterval 0 1$ de telle sorte que
    \begin{align}
        \sum_{I \in \Omega} \Pr(I) = 1
    \end{align}
\end{definition}

\end{document}
